%%%%%%%%%%%%%%%%%%%%%%%%%%%%%%%%%%%%%%%%%%%%%%%%%%%%%%%%%%%%%%%%%%%%%%%%%%%%%
%% INÍCIO DO GLOSSÁRIO                                                     %%
%%%%%%%%%%%%%%%%%%%%%%%%%%%%%%%%%%%%%%%%%%%%%%%%%%%%%%%%%%%%%%%%%%%%%%%%%%%%%

\chapter*{Glossário}
\addcontentsline{toc}{chapter}{Glossário}
\markboth{Glossário}{Glossário}

% ========================================================================= %
\section*{Acrônimos}

BFS    \dotfill {\it Breadth-First Search}

BGP    \dotfill {\it Border Gateway Protocol}

CAIDA  \dotfill {\it Cooperative Association for Internet Data Analysis}

CDF    \dotfill {\it Cumulative Distribution Function}

DDoS   \dotfill {\it Distributed Denial of Service}

DoS    \dotfill {\it Denial of Service}

FIFO   \dotfill {\it First-In First-Out}

IDS    \dotfill {\it Intrusion Detection System}

IoT    \dotfill {\it Internet of Things}

IP     \dotfill {\it Internet Protocol}

IPv4   \dotfill {\it Internet Protocol version 4}

IPv6   \dotfill {\it Internet Protocol version 6}

IPS    \dotfill {\it Intrusion Prevention System}

ISN    \dotfill {\it Initial Sequence Number}

NAPT   \dotfill {\it Network Address and Port Translation}

NAT    \dotfill {\it Network Address Translation}

NAT-PT \dotfill {\it Network Address Translation -- Protocol Translation}

NP     \dotfill {\it Nondeterministic Polynomial Time}

P2P    \dotfill {\it Peer to Peer}

PDF    \dotfill {\it Probability Distribution Function}

PRNG   \dotfill {\it Pseudo-Random Number Generator}

SOM    \dotfill {\it Self-Organizing Map}

TCP    \dotfill {\it Transmission Control Protocol}

% ========================================================================= %
%\section*{Grandezas}
%
%Definição de prefixos para múltiplos binários.
%
%\begin{center}
%\begin{tabular}{lcccccccc}
%{\bf Prefixo} & {\it kilo} & {\it mega} & {\it giga}  & {\it tera} &
%                {\it peta} & {\it exa}  & {\it zetta} & {\it yotta}
%                \tabularnewline
%{\bf Valor}   & $2^{10}$   & $2^{20}$   & $2^{30}$    & $2^{40}$   &
%                $2^{50}$   & $2^{60}$   & $2^{70}$    & $2^{80}$
%\end{tabular}
%\end{center}

% ========================================================================= %
\section*{Simbologia}

\begin{longtable}{p{1in}p{4.5in}}

% ------------------------------------------------------------------------- %
%\multicolumn{2}{l}{\bf Delimitadores}
%\tabularnewline
%\tabularnewline

\qedsymbol \dotfill &
\index{\qedsymbol}%
Demarcador contração de `como se queria demonstrar'.
\tabularnewline
\tabularnewline

$\square$ \dotfill &
\index{$\square$}%
Demarca fim de Algoritmos, Definições, Teoremas, dentre outros.
\tabularnewline
\tabularnewline

% ------------------------------------------------------------------------- %
\multicolumn{2}{l}{\bf Representações}
\tabularnewline
\tabularnewline

$\mathbf{x}$ \dotfill &
\index{vetor}%
Letras minúsculas em negrito indicam vetores coluna.
É possível parametrizar o vetor, por exemplo,
$\mathbf{x}(t)=
    \begin{bmatrix}x_1(t) & \dotsb & x_n(t)\end{bmatrix}^\intercal$
indica que o vetor $\mathbf{x}$ é variante no tempo.
\tabularnewline
\tabularnewline

$\mathbf{X}$ \dotfill &
\index{matriz}%
Letras maiúsculas em negrito indicam matrizes.
Assim como é possível parametrizar vetores, o mesmo é possível com matrizes,
por exemplo, uma matriz variante no tempo pode ser representa por
$\mathbf{X}(t)$.
\tabularnewline
\tabularnewline

$\mathcal{X}$ \dotfill &
\index{variavel aleatoria@variável aleatória}%
Letras maiúsculas caligráficas representam variáveis aleatórias.
\tabularnewline
\tabularnewline

$\dot{x}(t)$ \dotfill &
\index{derivada}%
Indica a derivada da função $x(\cdot)$ em relação ao tempo $t$.
Também se aplica a funcionais em vetores e matrizes.
\tabularnewline
\tabularnewline

$n!$ \dotfill &
\index{fatorial}%
Operador fatorial, definido recursivamente como $n!=n(n-1)!$ e com caso base
$0!=1$.
De forma iterativa também pode ser descrito como
\begin{equation*}
n!=\prod_{i=0}^{n-2} (n-i)\text{,}
\end{equation*}
para $n\geq 2$.
\tabularnewline
\tabularnewline

$\binom{n}{k}$ \dotfill &
\index{coeficiente binomial}%
Coeficiente binomial de $n$ dado $k$, onde $0 \leq k \leq n$, definido como
\begin{equation*}
\binom{n}{k} = \frac{n!}{k!(n-k)!}\text{,}
\end{equation*}
que pode ser computado de forma eficiente utilizando
\begin{equation*}
\binom{n}{k} = \prod_{i=1}^k \frac{n-(k-i)}{i}\text{,}
\end{equation*}
que possui complexidade $\Theta(k)$.
\tabularnewline
\tabularnewline

$\delta(t),\delta_{ij}$ \dotfill &
\index{delta de Kronecker}%
\index{$\delta(t),\delta_{ij}$}%
A função delta de Kronecker, definida como
\begin{equation*}
\delta_{ij} \triangleq
\left\{
\begin{array}{cl}
    1         & \text{se } i = j \\
    0         & \text{caso contrário}
\end{array}
\right.\text{,}
\end{equation*}
utilizada como contrapartida discreta da função delta de Dirac.
\index{delta de Dirac}%
Por conveniência, é possível usar a seguinte representação
\begin{equation*}
\delta(t) \triangleq
\left\{
\begin{array}{cl}
    1         & \text{se } t = 0 \\
    0         & \text{caso contrário}
\end{array}
\right.\text{.}
\end{equation*}
Dessa forma temos de forma equivalente que o valor $\delta(i - j)$ é $1$ se
$i=j$ e $0$ caso contrário.
\tabularnewline
\tabularnewline

$\operatorname{H}_n$ \dotfill &
\index{serie harmonica@série harmônica}%
\index{$\operatorname{H}_n$}%
Indica a soma dos $n$ primeiros termos da série harmônica, representada por
\begin{equation*}
\operatorname{H}_n = \sum_{i=1}^n \frac{1}{i}\text{,}
\end{equation*}
que diverge no limite quando $n \rightarrow \infty$.
Porém, possui a seguinte propriedade assintótica
\begin{equation*}
\lim_{n \rightarrow \infty}\operatorname{H}_n - \log(n) = \gamma\text{,}
\end{equation*}
onde $\gamma\approx 0.57721$ representa a constante de Euler-Mascheroni.
\index{constante de Euler-Mascheroni}%
Portanto, é possível usar a seguinte igualdade assintótica
\begin{equation*}
\operatorname{H}_n \simeq \log(n) + \gamma\text{,}
\end{equation*}
onde o logaritmo natural é o da base natural $e$.
\tabularnewline
\tabularnewline

$\{x : p(x)\}$ \dotfill &
Descrição do conjunto representado pelos elementos $x$ que têm a propriedade,
ou predicado, $p(x)$.
Adicionalmente, o predicado $p(x)$ pode ser descrito utilizando os operadores
da lógica proposicional.
\tabularnewline
\tabularnewline

$(\forall x)(p(x))$ \dotfill &
\index{quantificador!universal}%
Quantificação universal em relação aos elementos $x$ que têm a propriedade,
ou predicado, $p(x)$.
A pertinência dos elementos representados por $x$ também pode ser descrita de
forma explicita, por exemplo, $(\forall x \in \mathbb{N})(p(x))$.
Que expressa que todos os elementos do conjunto dos números naturais possuem
o predicado $p$.
Adicionalmente, o predicado $p(x)$ pode ser descrito utilizando os operadores
da lógica proposicional.
\tabularnewline
\tabularnewline

$(\exists x)(p(x))$ \dotfill &
\index{quantificador!existencial}%
Quantificação existencial em relação aos elementos $x$ que têm a propriedade,
ou predicado, $p(x)$.
A pertinência dos elementos representados por $x$ também pode ser descrita de
forma explicita, por exemplo, $(\exists x \in \mathbb{N})(p(x))$.
Que expressa que existe pelo menos um número natural que possui o predicado
$p$.
Adicionalmente, o predicado $p(x)$ pode ser descrito utilizando os operadores
da lógica proposicional.
\tabularnewline
\tabularnewline

% ------------------------------------------------------------------------- %
\multicolumn{2}{l}{\bf Notação assintótica}
\tabularnewline
\tabularnewline

$\operatorname{O}(\cdot)$ \dotfill &
\index{$\operatorname{O}(\cdot)$}%
Quando é expresso que $f(n) \in \operatorname{O}(g(n))$%
\footnote{Utiliza-se o símbolo de pertinência $\in$ pois interpreta-se que o
operador $\operatorname{O}(\cdot)$ representa o conjunto das funções que são
limitadas superiormente pelo seu argumento, no caso a função $g(\cdot)$.
O mesmo princípio pode ser aplicada aos outros operadores assintóticos
apresentados em sequência.}%
, dize-se que existe uma constante $k$, tal que a função $f(n)$, para todo
valor de $n>n_0$, é sempre limitada superiormente por $kg(n)$.
\tabularnewline
\tabularnewline

$\Omega(\cdot)$ \dotfill &
\index{$\Omega(\cdot)$}%
Quando é expresso que $f(n)\in\Omega(g(n))$, dize-se que existe uma constante
$k$, tal que a função $f(n)$, para todo valor de $n>n_0$, é sempre limitada
inferiormente por $kg(n)$.
\tabularnewline
\tabularnewline

$\Theta(\cdot)$ \dotfill &
\index{$\Theta(\cdot)$}%
Quando é expresso que $f(n)\in\Theta(g(n))$, dize-se que existe uma constante
$k_1$, tal que a função $f(n)$, para todo valor de $n>n_0$, é sempre limitada
inferiormente por $k_1g(n)$, e também existe uma outra constante $k_2$, tal
que a função $f(n)$, para todo valor de $n>n_0$, é sempre limitada
superiormente por $k_2g(n)$.
De forma equivalente, define-se que $f(n)\in\Theta(g(n))$ se e somente se
\begin{equation*}
\lim_{n\rightarrow\infty}\frac{f(n)}{g(n)}=c\text{,}
\end{equation*}
para $g(n)$ diferente de zero ou, pelo menos, sempre maior de que zero a
partir de algum ponto e para $0<c<\infty$.
\tabularnewline
\tabularnewline

% ------------------------------------------------------------------------- %
\multicolumn{2}{l}{\bf Igualdades matemáticas}
\index{igualdades}
\tabularnewline
\tabularnewline

$\approx$ \dotfill &
\index{$\approx$}%
Valor aproximado.
\tabularnewline
\tabularnewline

$\simeq$ \dotfill &
\index{$\simeq$}%
Igualdade assintótica, isto é, se $f(n) \simeq g(n)$ então
\begin{equation*}
\lim_{n\rightarrow\infty}\frac{f(n)}{g(n)}=1\text{,}
\end{equation*}
para $g(\cdot)$ infinitamente diferente de zero.
\tabularnewline
\tabularnewline

$\propto$ \dotfill &
\index{$\propto$}%
Proporcionalidade, isto é, se $f(n) \propto g(n)$, então existe uma constante
$k$ tal que $f(n)=kg(n)$.
De forma generalista, pode considerar também a igualdade assintótica.
\tabularnewline
\tabularnewline

$\triangleq$ \dotfill &
\index{$\triangleq$}%
Igualdade por definição, por exemplo,
\begin{equation*}
\frac{d\mathbf{x}(t)}{dt}
\triangleq
\begin{bmatrix}
\frac{dx_1(t)}{dt} & \dotsb & \frac{dx_n(t)}{dt}
\end{bmatrix}^\intercal\text{,}
\end{equation*}
onde $\mathbf{x}(t)$ é um vetor coluna.
\tabularnewline
\tabularnewline

$\equiv$ \dotfill &
\index{$\equiv$}%
Equivalência, por exemplo, $x \equiv y$ significa que $x$ é definido como
sendo logicamente igual à $y$.
\tabularnewline
\tabularnewline

% ------------------------------------------------------------------------- %
\multicolumn{2}{l}{\bf Notação estatística}
\tabularnewline
\tabularnewline

$\sim$ \dotfill &
Indicador de distribuição de probabilidade, por exemplo $\mathcal{X} \sim
N(\mu,\sigma)$ indica que a variável aleatória $\mathcal{X}$ segue uma
distribuição de probabilidade normal com média $\mu$ e desvio padrão
$\sigma$.
\index{media@média}%
\index{desvio padrao@desvio padrão}%
\tabularnewline
\tabularnewline

$\mathcal{X}_\zeta$ \dotfill &
\index{variavel aleatoria@variável aleatória!realizacao@realização}%
Resultado ou realização $\zeta$ da variável aleatória $\mathcal{X}$.
\tabularnewline
\tabularnewline

$\operatorname{P}(\mathcal{X}_\zeta)$ \dotfill &
\index{probabilidade}%
\index{$\operatorname{P}(\mathcal{X}_\zeta)$}%
Probabilidade da variável aleatória $\mathcal{X}$ assumir a realização
$\zeta$.
\tabularnewline
\tabularnewline

$\operatorname{P}(\mathcal{X}_\zeta\mid p)$ \dotfill &
\index{probabilidade!condicional}%
\index{$\operatorname{P}(\mathcal{X}_\zeta\mid p)$}%
Probabilidade da variável aleatória $\mathcal{X}$ assumir a realização
$\zeta$ dado que o predicado $p$ é verdadeiro.
\tabularnewline
\tabularnewline

$\operatorname{E}\{\mathcal{X}\}$ \dotfill &
\index{valor esperado}%
\index{$\operatorname{E}\{\mathcal{X}\}$}%
Valor esperado da variável aleatória $\mathcal{X}$.
No caso discreto é definido como
\begin{equation*}
\operatorname{E}\{\mathcal{X}\}=
    \sum_{\{\zeta \in \mho\}}
        \mathcal{X}_\zeta \operatorname{P}(\mathcal{X}_\zeta)\text{,}
\end{equation*}
onde $\mho$ é o conjunto de possíveis realizações da variável aleatória.
\tabularnewline
\tabularnewline

$\operatorname{E}\{\mathcal{X}\mid p\}$ \dotfill &
Valor esperado da variável aleatória $\mathcal{X}$ dado que o predicado $p$
é verdadeiro.
No caso discreto é definido como
\begin{equation*}
\operatorname{E}\{\mathcal{X}\}=
    \sum_{\{\zeta \in \mho\}}
        \mathcal{X}_\zeta \operatorname{P}(\mathcal{X}_\zeta\mid p)\text{,}
\end{equation*}
onde $\mho$ é o conjunto de possíveis realizações da variável aleatória.
\tabularnewline
\tabularnewline

% ------------------------------------------------------------------------- %
\multicolumn{2}{l}{\bf Operadores matemáticos}
\tabularnewline
\tabularnewline

$|\cdot|$ \dotfill &
\index{valor absoluto}%
\index{cardinalidade}%
Se for aplicado a um escalar, indica o seu valor absoluto.
Caso seja aplicado a um conjunto, indica sua cardinalidade.
\tabularnewline
\tabularnewline

$\lfloor\cdot\rfloor$ \dotfill &
O maior valor inteiro menor ou igual ao escalar.
\tabularnewline
\tabularnewline

$\lceil\cdot\rceil$ \dotfill &
O menor valor inteiro maior ou igual ao escalar.
\tabularnewline
\tabularnewline

$\rho(\cdot)$ \dotfill &
\index{$\rho(\cdot)$}%
\index{posto}%
\index{matriz!posto da}%
Posto de uma matriz, por exemplo dada uma matriz identidade
$\mathbf{I}_{n \times n}$, $\rho(\mathbf{I})=n$.
\tabularnewline
\tabularnewline

$\mathbf{X}^\intercal$ \dotfill &
\index{matriz!transposta}%
Operação de transposição da matriz $\mathbf{X}$, isto é, troca dos elementos
$x_{ij}$ pelos elementos $x_{ji}$.
Também pode ser aplicada a vetores, no qual transforma vetores coluna em
vetores linha, e vice-versa.
\tabularnewline
\tabularnewline

$X - Y$ \dotfill &
Subtração de elementos de conjuntos.
Utilizando a notação de conjuntos pode ser definido por
\begin{equation*}
X - Y \triangleq
    \{z : (z \in X) \wedge (z \notin Y)\}\text{,}
\end{equation*}
que representa o conjunto resultante da retirada dos elementos em $X$ que
também estão em $Y$.
\tabularnewline
\tabularnewline

$X \times Y$ \dotfill &
\index{produto cartesiano}%
Produto cartesiano entre dois conjuntos $X$ e $Y$.
Utilizando a notação de conjuntos pode ser definido por
\begin{equation*}
X \times Y \triangleq
    \{(x,y) : (x \in X) \wedge (y \in Y)\}\text{,}
\end{equation*}
que representa todas as possíveis combinações de pares ordenados entres os
elementos de $X$ e de $Y$.
\tabularnewline
\tabularnewline

% ------------------------------------------------------------------------- %
\multicolumn{2}{l}{\bf Operadores lógicos}
\tabularnewline
\tabularnewline

$\neg$ \dotfill &
\index{negacao@negação}%
Operador unário de negação.
\tabularnewline
\tabularnewline

$\vee$ \dotfill &
\index{disjuncao@disjunção}%
Operador binário de disjunção, definido como `ou inclusivo'.
\tabularnewline
\tabularnewline

$\wedge$ \dotfill &
\index{conjuncao@conjunção}%
Operador binário de conjunção, definido com valor lógico `e'.
\tabularnewline
\tabularnewline

$\Rightarrow$ \dotfill &
\index{implicacao@implicação}%
Operador binário de implicação, por exemplo, $(a \Rightarrow b)$, onde $a$ é
denominado antecedente e $b$ consequente.
Único operador binário não comutativo.
\tabularnewline
\tabularnewline

$\Leftrightarrow$ \dotfill &
\index{bi-implicacao@bi-implicação}%
Operador binário de bi-implicação.
Onde $(a \Leftrightarrow b)$ é logicamente equivalente a representação
$((a \Rightarrow b) \wedge (b \Rightarrow a))$.

\end{longtable}


%%%%%%%%%%%%%%%%%%%%%%%%%%%%%%%%%%%%%%%%%%%%%%%%%%%%%%%%%%%%%%%%%%%%%%%%%%%%%
%% FIM DO GLOSSÁRIO                                                        %%
%%%%%%%%%%%%%%%%%%%%%%%%%%%%%%%%%%%%%%%%%%%%%%%%%%%%%%%%%%%%%%%%%%%%%%%%%%%%%
